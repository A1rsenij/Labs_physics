\documentclass[a4paper, 12pt]{article}

\usepackage{hyperref}
\usepackage[warn]{mathtext}
\usepackage[utf8]{inputenc}
\usepackage[T2A]{fontenc}
\usepackage[english,russian]{babel}
\usepackage{multirow}
\usepackage{amsmath,amsfonts,amssymb,amsthm,mathtools}
\usepackage{indentfirst}
\DeclareSymbolFont{T2Aletters}{T2A}{cmr}{m}{it}
\usepackage{ gensymb }
\mathtoolsset{showonlyrefs=true}
\usepackage{euscript}
\usepackage{mathrsfs}
\usepackage[left=2cm,right=2cm,top=2cm,bottom=2cm]{geometry}
\usepackage{graphicx}
\usepackage{wrapfig}
\usepackage[rgb]{xcolor}
\hypersetup{
colorlinks=true,
urlcolor=blue
}


\title{Лабораторная работа}
\author{Гисич Арсений Б03-109}
\date{2022}

\begin{document}

	\begin{center}
		{\large МОСКОВСКИЙ ФИЗИКО-ТЕХНИЧЕСКИЙ ИНСТИТУТ (НАЦИОНАЛЬНЫЙ ИССЛЕДОВАТЕЛЬСКИЙ УНИВЕРСИТЕТ)}
	\end{center}
	\vspace{5 cm}
	{\Large
		\begin{center}
			{\bf Лабораторная работа 2.1.3}\\[0.2 cm]
			Определение $C_p/C_v$ по скорости звука в газе
		\end{center}
	}
	\vspace{4 cm}
	\begin{flushright}
		{\Large Выполнил: \\
			\vspace{0.2 cm}
			Гисич Арсений \\
			\vspace{0.2 cm}
			Б03-109 \\}
	\end{flushright}
	\vspace{9 cm}
	\begin{center}
		Долгопрудный\\[0.1 cm]
		2022
	\end{center}
\thispagestyle{empty}

\section{Аннотация}

\par Цель работы: 1) измерение частоты колебаний и длины волны при резонансе звуковых колебаний в газе, заполняющем трубу; 2) определение показателя адиабаты с помощью уравнения состояния идеального газа.

\section{Теоретические сведения}

Скорость распространения звуковой волны в газах зависит от показателя адиабаты $ \gamma $. На измерении скорости звука основан один из наиболее точных методов определения показателя адиабаты.

Скорость звука в газах определяется формулой:

\begin{equation}\label{velocity}
c=\sqrt{\gamma\frac{RT}{\mu}}.
\end{equation}
где $ R $ -- газовая постоянная, $ T $ -- температура газа, а $ \mu $ -- его молярная масса. Преобразуя эту формулу, найдем
\begin{equation}\label{gamma}
\boxed{\gamma = \frac{\mu}{RT}c^2}.
\end{equation}

Таким образом, для определения показателя адиабаты достаточно измерить температуру газа и скорость распространения звука (молярная масса газа предполагается известной).

Звуковая волна, распространяющаяся вдоль трубы, испытывает многократные отражения от торцов. Звуковые колебания в трубе являются наложением всех отраженных волн и очень сложны. Картина упрощается, если длина трубы $ L $ равна целому числу полуволн, то есть когда \[ L=n\lambda/2, \] где $ \lambda $ -- длина волны звука в трубе, а $ n $ -- любое целое число. Если это условие выполнено, то волна, отраженная от торца трубы, вернувшаяся к ее началу и вновь отраженная, совпадает по фазе с падающей. Совпадающие по фазе волны усиливают друг друга. Амплитуда звуковых колебаний при этом резко возрастает -- наступает резонанс.

При звуковых колебаниях слои газа, прилегающие к торцам трубы, не испытывают смещения. Узлы смещения повторяются по всей длине трубы через $ \lambda/2 $. Между узлами находятся максимумы смещения.

Скорость звука c связана с его частотой $ f $ и длиной волны $ \lambda $ соотношением

\begin{equation}\label{lambda_f}
c=\lambda f.
\end{equation}

Подбор условий, при которых возникает резонанс, можно производить двояко:
\begin{enumerate}
	\item При неизменной частоте $ f $ звукового генератора (а следовательно, и неизменной длине звуковой волны $ \lambda $) можно изменять длину трубы $ L $. Для этого применяется раздвижная труба. Длина раздвижной трубы постепенно увеличивается, и наблюдается ряд последовательных резонансов. Возникновение резонанса легко наблюдать на осциллографе по резкому увеличению амплитуды колебаний. Для последовательных резонансов имеем \begin{equation}\label{first}
	L_n=n\frac{\lambda}{2}, \quad L_{n+1}=(n+1)\frac{\lambda}{2}, \quad \dots, \quad L_{n+k} = n\frac{\lambda}{2}+k\frac{\lambda}{2},
	\end{equation} т. е. $ \lambda/2 $ равно угловому коэффициенту графика, изображающего зависимость длины трубы $ L $ от номера резонанса $ k $. Скорость звука находится по формуле \eqref{lambda_f}.
	\item При постоянной длине трубы можно изменять частоту звуковых колебаний. В этом случае следует плавно изменять частоту $ f $ звукового генератора, а следовательно, и длину звуковой волны $ \lambda $. Для последовательных резонансов получим 
	\begin{equation}\label{4}
	L=\frac{\lambda_1}{2}n=\frac{\lambda_2}{2}(n+1)=\dots=\frac{\lambda_{k+1}}{2}(n+k).
	\end{equation}
	
	Из \eqref{lambda_f} и \eqref{4} имеем:
	\[ f_1=\frac{c}{\lambda_1}=\frac{c}{2L}n, \quad f_2=\frac{c}{\lambda_2}=\frac{c}{2L}(n+1)=f_1+\frac{c}{2L},\quad \dots, \]
	\begin{equation}\label{5}
	f_{k+1}=\frac{c}{\lambda_{k+1}}=\frac{c}{2L}(n+k)=f_1+\frac{c}{2L}k.
	\end{equation}
	Скорость звука, деленная на $ 2L $, определяется, таким образом, по угловому коэффициенту графика зависимости частоты от номера резонанса.
\end{enumerate}

\section{Методика измерений}

\begin{figure}[h!]
	\begin{center}
		\includegraphics[width=12cm]{ust1.jpg}
	\end{center}
	\caption{Установка для измерения скорости звука при помощи раздвижной трубы.}
	\label{img1}
\end{figure}

\begin{figure}[h!]
	\begin{center}
		\includegraphics[width=12cm]{ust2.jpg}
	\end{center}
	\caption{Установка для изучения зависимости скорости звука от температуры.}
	\label{img2}
\end{figure}

Соответственно двум методам измерения скорости звука в работе имеются две установки (рис. \ref{img1} и \ref{img2}). В обеих установках звуковые колебания в трубе возбуждаются телефоном Т и улавливаются микрофоном М. Мембрана телефона приводится в движение переменным током звуковой частоты; в качестве источника переменной ЭДС используется звуковой генератор ГЗ. Возникающий в микрофоне сигнал наблюдается на осциллографе ЭО.

Микрофон и телефон присоединены к установке через тонкие резиновые трубки. Такая связь достаточна для возбуждения и обнаружения звуковых колебаний в трубе и в то же время мало возмущает эти колебания: при расчетах оба торца трубы можно считать неподвижными, а влиянием соединительных отверстий пренебречь.

Первая установка (рис. \ref{img1}) содержит раздвижную трубу с миллиметровой шкалой. Через патрубок (на рисунке не показан) труба может наполняться воздухом или углекислым газом из газгольдера. На этой установке производятся измерения $ \gamma $ для воздуха и для $ CO_2 $. Вторая установка (рис. \ref{img2}) содержит теплоизолированную трубу постоянной длины. Воздух в трубе нагревается водой из термостата. Температура газа принимается равной температуре омывающей трубу воды. На этой установке измеряется зависимость скорости звука от температуры.

\section{Используемое оборудование}

\begin{enumerate}
    \item звуковой генератор ГЗ;
    \item электронный осциллограф ЭО;
    \item микрофон;
    \item телефон;
    \item раздвижная труба;
    \item теплоизолированная труба, обогреваемая водой из термостата;
    \item баллон со сжатым углекислым газом;
    \item газгольдер.
\end{enumerate}

\section{Результаты измерений и обработка данных}

Начальные условия и параметры установки:

$\begin{aligned}
& P_{\text{атм}} = 101,1\pm0,05~кПа\\
& V_{K5+K6+кап} = 50~см^3 \\
& L = 10,8~см \\
& d_{кап} = 0,8~мм \\
& \rho_{масла} = 0.885~г/см^3
\end{aligned}$\\[0,5 cm]

\section{Обсуждение результатов и выводы}

В данной работе исследовалась зависимость давления в установке от времени. По результатам измерения давления различными способами определялась производительность вакуумного насоса. Полученное значение для скорости откачки:
\[\boxed{W = 0,258\pm0,012~л/с}.\]
Использованный в работе метод измерений позволяет достичь относительной точности результатов в 5\%. Метод расчёта скорости откачки по зависимости давления от времени при улучшении вакуума оказался точнее в сравнении с методом расчёта по различию $P_{уст}$ и $P_{пр}$. Основной вклад в погрешность вносит погрешность определения коэффициентов линейной аппроксимации.
Также в данной работе были проверены теоретические зависимости, связанные с течением газа.

\end{document}